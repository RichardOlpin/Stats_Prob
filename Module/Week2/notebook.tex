
% Default to the notebook output style

    


% Inherit from the specified cell style.




    
\documentclass[11pt]{article}

    
    
    \usepackage[T1]{fontenc}
    % Nicer default font (+ math font) than Computer Modern for most use cases
    \usepackage{mathpazo}

    % Basic figure setup, for now with no caption control since it's done
    % automatically by Pandoc (which extracts ![](path) syntax from Markdown).
    \usepackage{graphicx}
    % We will generate all images so they have a width \maxwidth. This means
    % that they will get their normal width if they fit onto the page, but
    % are scaled down if they would overflow the margins.
    \makeatletter
    \def\maxwidth{\ifdim\Gin@nat@width>\linewidth\linewidth
    \else\Gin@nat@width\fi}
    \makeatother
    \let\Oldincludegraphics\includegraphics
    % Set max figure width to be 80% of text width, for now hardcoded.
    \renewcommand{\includegraphics}[1]{\Oldincludegraphics[width=.8\maxwidth]{#1}}
    % Ensure that by default, figures have no caption (until we provide a
    % proper Figure object with a Caption API and a way to capture that
    % in the conversion process - todo).
    \usepackage{caption}
    \DeclareCaptionLabelFormat{nolabel}{}
    \captionsetup{labelformat=nolabel}

    \usepackage{adjustbox} % Used to constrain images to a maximum size 
    \usepackage{xcolor} % Allow colors to be defined
    \usepackage{enumerate} % Needed for markdown enumerations to work
    \usepackage{geometry} % Used to adjust the document margins
    \usepackage{amsmath} % Equations
    \usepackage{amssymb} % Equations
    \usepackage{textcomp} % defines textquotesingle
    % Hack from http://tex.stackexchange.com/a/47451/13684:
    \AtBeginDocument{%
        \def\PYZsq{\textquotesingle}% Upright quotes in Pygmentized code
    }
    \usepackage{upquote} % Upright quotes for verbatim code
    \usepackage{eurosym} % defines \euro
    \usepackage[mathletters]{ucs} % Extended unicode (utf-8) support
    \usepackage[utf8x]{inputenc} % Allow utf-8 characters in the tex document
    \usepackage{fancyvrb} % verbatim replacement that allows latex
    \usepackage{grffile} % extends the file name processing of package graphics 
                         % to support a larger range 
    % The hyperref package gives us a pdf with properly built
    % internal navigation ('pdf bookmarks' for the table of contents,
    % internal cross-reference links, web links for URLs, etc.)
    \usepackage{hyperref}
    \usepackage{longtable} % longtable support required by pandoc >1.10
    \usepackage{booktabs}  % table support for pandoc > 1.12.2
    \usepackage[inline]{enumitem} % IRkernel/repr support (it uses the enumerate* environment)
    \usepackage[normalem]{ulem} % ulem is needed to support strikethroughs (\sout)
                                % normalem makes italics be italics, not underlines
    

    
    
    % Colors for the hyperref package
    \definecolor{urlcolor}{rgb}{0,.145,.698}
    \definecolor{linkcolor}{rgb}{.71,0.21,0.01}
    \definecolor{citecolor}{rgb}{.12,.54,.11}

    % ANSI colors
    \definecolor{ansi-black}{HTML}{3E424D}
    \definecolor{ansi-black-intense}{HTML}{282C36}
    \definecolor{ansi-red}{HTML}{E75C58}
    \definecolor{ansi-red-intense}{HTML}{B22B31}
    \definecolor{ansi-green}{HTML}{00A250}
    \definecolor{ansi-green-intense}{HTML}{007427}
    \definecolor{ansi-yellow}{HTML}{DDB62B}
    \definecolor{ansi-yellow-intense}{HTML}{B27D12}
    \definecolor{ansi-blue}{HTML}{208FFB}
    \definecolor{ansi-blue-intense}{HTML}{0065CA}
    \definecolor{ansi-magenta}{HTML}{D160C4}
    \definecolor{ansi-magenta-intense}{HTML}{A03196}
    \definecolor{ansi-cyan}{HTML}{60C6C8}
    \definecolor{ansi-cyan-intense}{HTML}{258F8F}
    \definecolor{ansi-white}{HTML}{C5C1B4}
    \definecolor{ansi-white-intense}{HTML}{A1A6B2}

    % commands and environments needed by pandoc snippets
    % extracted from the output of `pandoc -s`
    \providecommand{\tightlist}{%
      \setlength{\itemsep}{0pt}\setlength{\parskip}{0pt}}
    \DefineVerbatimEnvironment{Highlighting}{Verbatim}{commandchars=\\\{\}}
    % Add ',fontsize=\small' for more characters per line
    \newenvironment{Shaded}{}{}
    \newcommand{\KeywordTok}[1]{\textcolor[rgb]{0.00,0.44,0.13}{\textbf{{#1}}}}
    \newcommand{\DataTypeTok}[1]{\textcolor[rgb]{0.56,0.13,0.00}{{#1}}}
    \newcommand{\DecValTok}[1]{\textcolor[rgb]{0.25,0.63,0.44}{{#1}}}
    \newcommand{\BaseNTok}[1]{\textcolor[rgb]{0.25,0.63,0.44}{{#1}}}
    \newcommand{\FloatTok}[1]{\textcolor[rgb]{0.25,0.63,0.44}{{#1}}}
    \newcommand{\CharTok}[1]{\textcolor[rgb]{0.25,0.44,0.63}{{#1}}}
    \newcommand{\StringTok}[1]{\textcolor[rgb]{0.25,0.44,0.63}{{#1}}}
    \newcommand{\CommentTok}[1]{\textcolor[rgb]{0.38,0.63,0.69}{\textit{{#1}}}}
    \newcommand{\OtherTok}[1]{\textcolor[rgb]{0.00,0.44,0.13}{{#1}}}
    \newcommand{\AlertTok}[1]{\textcolor[rgb]{1.00,0.00,0.00}{\textbf{{#1}}}}
    \newcommand{\FunctionTok}[1]{\textcolor[rgb]{0.02,0.16,0.49}{{#1}}}
    \newcommand{\RegionMarkerTok}[1]{{#1}}
    \newcommand{\ErrorTok}[1]{\textcolor[rgb]{1.00,0.00,0.00}{\textbf{{#1}}}}
    \newcommand{\NormalTok}[1]{{#1}}
    
    % Additional commands for more recent versions of Pandoc
    \newcommand{\ConstantTok}[1]{\textcolor[rgb]{0.53,0.00,0.00}{{#1}}}
    \newcommand{\SpecialCharTok}[1]{\textcolor[rgb]{0.25,0.44,0.63}{{#1}}}
    \newcommand{\VerbatimStringTok}[1]{\textcolor[rgb]{0.25,0.44,0.63}{{#1}}}
    \newcommand{\SpecialStringTok}[1]{\textcolor[rgb]{0.73,0.40,0.53}{{#1}}}
    \newcommand{\ImportTok}[1]{{#1}}
    \newcommand{\DocumentationTok}[1]{\textcolor[rgb]{0.73,0.13,0.13}{\textit{{#1}}}}
    \newcommand{\AnnotationTok}[1]{\textcolor[rgb]{0.38,0.63,0.69}{\textbf{\textit{{#1}}}}}
    \newcommand{\CommentVarTok}[1]{\textcolor[rgb]{0.38,0.63,0.69}{\textbf{\textit{{#1}}}}}
    \newcommand{\VariableTok}[1]{\textcolor[rgb]{0.10,0.09,0.49}{{#1}}}
    \newcommand{\ControlFlowTok}[1]{\textcolor[rgb]{0.00,0.44,0.13}{\textbf{{#1}}}}
    \newcommand{\OperatorTok}[1]{\textcolor[rgb]{0.40,0.40,0.40}{{#1}}}
    \newcommand{\BuiltInTok}[1]{{#1}}
    \newcommand{\ExtensionTok}[1]{{#1}}
    \newcommand{\PreprocessorTok}[1]{\textcolor[rgb]{0.74,0.48,0.00}{{#1}}}
    \newcommand{\AttributeTok}[1]{\textcolor[rgb]{0.49,0.56,0.16}{{#1}}}
    \newcommand{\InformationTok}[1]{\textcolor[rgb]{0.38,0.63,0.69}{\textbf{\textit{{#1}}}}}
    \newcommand{\WarningTok}[1]{\textcolor[rgb]{0.38,0.63,0.69}{\textbf{\textit{{#1}}}}}
    
    
    % Define a nice break command that doesn't care if a line doesn't already
    % exist.
    \def\br{\hspace*{\fill} \\* }
    % Math Jax compatability definitions
    \def\gt{>}
    \def\lt{<}
    % Document parameters
    \title{Homework Week 2}
    
    
    

    % Pygments definitions
    
\makeatletter
\def\PY@reset{\let\PY@it=\relax \let\PY@bf=\relax%
    \let\PY@ul=\relax \let\PY@tc=\relax%
    \let\PY@bc=\relax \let\PY@ff=\relax}
\def\PY@tok#1{\csname PY@tok@#1\endcsname}
\def\PY@toks#1+{\ifx\relax#1\empty\else%
    \PY@tok{#1}\expandafter\PY@toks\fi}
\def\PY@do#1{\PY@bc{\PY@tc{\PY@ul{%
    \PY@it{\PY@bf{\PY@ff{#1}}}}}}}
\def\PY#1#2{\PY@reset\PY@toks#1+\relax+\PY@do{#2}}

\expandafter\def\csname PY@tok@w\endcsname{\def\PY@tc##1{\textcolor[rgb]{0.73,0.73,0.73}{##1}}}
\expandafter\def\csname PY@tok@c\endcsname{\let\PY@it=\textit\def\PY@tc##1{\textcolor[rgb]{0.25,0.50,0.50}{##1}}}
\expandafter\def\csname PY@tok@cp\endcsname{\def\PY@tc##1{\textcolor[rgb]{0.74,0.48,0.00}{##1}}}
\expandafter\def\csname PY@tok@k\endcsname{\let\PY@bf=\textbf\def\PY@tc##1{\textcolor[rgb]{0.00,0.50,0.00}{##1}}}
\expandafter\def\csname PY@tok@kp\endcsname{\def\PY@tc##1{\textcolor[rgb]{0.00,0.50,0.00}{##1}}}
\expandafter\def\csname PY@tok@kt\endcsname{\def\PY@tc##1{\textcolor[rgb]{0.69,0.00,0.25}{##1}}}
\expandafter\def\csname PY@tok@o\endcsname{\def\PY@tc##1{\textcolor[rgb]{0.40,0.40,0.40}{##1}}}
\expandafter\def\csname PY@tok@ow\endcsname{\let\PY@bf=\textbf\def\PY@tc##1{\textcolor[rgb]{0.67,0.13,1.00}{##1}}}
\expandafter\def\csname PY@tok@nb\endcsname{\def\PY@tc##1{\textcolor[rgb]{0.00,0.50,0.00}{##1}}}
\expandafter\def\csname PY@tok@nf\endcsname{\def\PY@tc##1{\textcolor[rgb]{0.00,0.00,1.00}{##1}}}
\expandafter\def\csname PY@tok@nc\endcsname{\let\PY@bf=\textbf\def\PY@tc##1{\textcolor[rgb]{0.00,0.00,1.00}{##1}}}
\expandafter\def\csname PY@tok@nn\endcsname{\let\PY@bf=\textbf\def\PY@tc##1{\textcolor[rgb]{0.00,0.00,1.00}{##1}}}
\expandafter\def\csname PY@tok@ne\endcsname{\let\PY@bf=\textbf\def\PY@tc##1{\textcolor[rgb]{0.82,0.25,0.23}{##1}}}
\expandafter\def\csname PY@tok@nv\endcsname{\def\PY@tc##1{\textcolor[rgb]{0.10,0.09,0.49}{##1}}}
\expandafter\def\csname PY@tok@no\endcsname{\def\PY@tc##1{\textcolor[rgb]{0.53,0.00,0.00}{##1}}}
\expandafter\def\csname PY@tok@nl\endcsname{\def\PY@tc##1{\textcolor[rgb]{0.63,0.63,0.00}{##1}}}
\expandafter\def\csname PY@tok@ni\endcsname{\let\PY@bf=\textbf\def\PY@tc##1{\textcolor[rgb]{0.60,0.60,0.60}{##1}}}
\expandafter\def\csname PY@tok@na\endcsname{\def\PY@tc##1{\textcolor[rgb]{0.49,0.56,0.16}{##1}}}
\expandafter\def\csname PY@tok@nt\endcsname{\let\PY@bf=\textbf\def\PY@tc##1{\textcolor[rgb]{0.00,0.50,0.00}{##1}}}
\expandafter\def\csname PY@tok@nd\endcsname{\def\PY@tc##1{\textcolor[rgb]{0.67,0.13,1.00}{##1}}}
\expandafter\def\csname PY@tok@s\endcsname{\def\PY@tc##1{\textcolor[rgb]{0.73,0.13,0.13}{##1}}}
\expandafter\def\csname PY@tok@sd\endcsname{\let\PY@it=\textit\def\PY@tc##1{\textcolor[rgb]{0.73,0.13,0.13}{##1}}}
\expandafter\def\csname PY@tok@si\endcsname{\let\PY@bf=\textbf\def\PY@tc##1{\textcolor[rgb]{0.73,0.40,0.53}{##1}}}
\expandafter\def\csname PY@tok@se\endcsname{\let\PY@bf=\textbf\def\PY@tc##1{\textcolor[rgb]{0.73,0.40,0.13}{##1}}}
\expandafter\def\csname PY@tok@sr\endcsname{\def\PY@tc##1{\textcolor[rgb]{0.73,0.40,0.53}{##1}}}
\expandafter\def\csname PY@tok@ss\endcsname{\def\PY@tc##1{\textcolor[rgb]{0.10,0.09,0.49}{##1}}}
\expandafter\def\csname PY@tok@sx\endcsname{\def\PY@tc##1{\textcolor[rgb]{0.00,0.50,0.00}{##1}}}
\expandafter\def\csname PY@tok@m\endcsname{\def\PY@tc##1{\textcolor[rgb]{0.40,0.40,0.40}{##1}}}
\expandafter\def\csname PY@tok@gh\endcsname{\let\PY@bf=\textbf\def\PY@tc##1{\textcolor[rgb]{0.00,0.00,0.50}{##1}}}
\expandafter\def\csname PY@tok@gu\endcsname{\let\PY@bf=\textbf\def\PY@tc##1{\textcolor[rgb]{0.50,0.00,0.50}{##1}}}
\expandafter\def\csname PY@tok@gd\endcsname{\def\PY@tc##1{\textcolor[rgb]{0.63,0.00,0.00}{##1}}}
\expandafter\def\csname PY@tok@gi\endcsname{\def\PY@tc##1{\textcolor[rgb]{0.00,0.63,0.00}{##1}}}
\expandafter\def\csname PY@tok@gr\endcsname{\def\PY@tc##1{\textcolor[rgb]{1.00,0.00,0.00}{##1}}}
\expandafter\def\csname PY@tok@ge\endcsname{\let\PY@it=\textit}
\expandafter\def\csname PY@tok@gs\endcsname{\let\PY@bf=\textbf}
\expandafter\def\csname PY@tok@gp\endcsname{\let\PY@bf=\textbf\def\PY@tc##1{\textcolor[rgb]{0.00,0.00,0.50}{##1}}}
\expandafter\def\csname PY@tok@go\endcsname{\def\PY@tc##1{\textcolor[rgb]{0.53,0.53,0.53}{##1}}}
\expandafter\def\csname PY@tok@gt\endcsname{\def\PY@tc##1{\textcolor[rgb]{0.00,0.27,0.87}{##1}}}
\expandafter\def\csname PY@tok@err\endcsname{\def\PY@bc##1{\setlength{\fboxsep}{0pt}\fcolorbox[rgb]{1.00,0.00,0.00}{1,1,1}{\strut ##1}}}
\expandafter\def\csname PY@tok@kc\endcsname{\let\PY@bf=\textbf\def\PY@tc##1{\textcolor[rgb]{0.00,0.50,0.00}{##1}}}
\expandafter\def\csname PY@tok@kd\endcsname{\let\PY@bf=\textbf\def\PY@tc##1{\textcolor[rgb]{0.00,0.50,0.00}{##1}}}
\expandafter\def\csname PY@tok@kn\endcsname{\let\PY@bf=\textbf\def\PY@tc##1{\textcolor[rgb]{0.00,0.50,0.00}{##1}}}
\expandafter\def\csname PY@tok@kr\endcsname{\let\PY@bf=\textbf\def\PY@tc##1{\textcolor[rgb]{0.00,0.50,0.00}{##1}}}
\expandafter\def\csname PY@tok@bp\endcsname{\def\PY@tc##1{\textcolor[rgb]{0.00,0.50,0.00}{##1}}}
\expandafter\def\csname PY@tok@fm\endcsname{\def\PY@tc##1{\textcolor[rgb]{0.00,0.00,1.00}{##1}}}
\expandafter\def\csname PY@tok@vc\endcsname{\def\PY@tc##1{\textcolor[rgb]{0.10,0.09,0.49}{##1}}}
\expandafter\def\csname PY@tok@vg\endcsname{\def\PY@tc##1{\textcolor[rgb]{0.10,0.09,0.49}{##1}}}
\expandafter\def\csname PY@tok@vi\endcsname{\def\PY@tc##1{\textcolor[rgb]{0.10,0.09,0.49}{##1}}}
\expandafter\def\csname PY@tok@vm\endcsname{\def\PY@tc##1{\textcolor[rgb]{0.10,0.09,0.49}{##1}}}
\expandafter\def\csname PY@tok@sa\endcsname{\def\PY@tc##1{\textcolor[rgb]{0.73,0.13,0.13}{##1}}}
\expandafter\def\csname PY@tok@sb\endcsname{\def\PY@tc##1{\textcolor[rgb]{0.73,0.13,0.13}{##1}}}
\expandafter\def\csname PY@tok@sc\endcsname{\def\PY@tc##1{\textcolor[rgb]{0.73,0.13,0.13}{##1}}}
\expandafter\def\csname PY@tok@dl\endcsname{\def\PY@tc##1{\textcolor[rgb]{0.73,0.13,0.13}{##1}}}
\expandafter\def\csname PY@tok@s2\endcsname{\def\PY@tc##1{\textcolor[rgb]{0.73,0.13,0.13}{##1}}}
\expandafter\def\csname PY@tok@sh\endcsname{\def\PY@tc##1{\textcolor[rgb]{0.73,0.13,0.13}{##1}}}
\expandafter\def\csname PY@tok@s1\endcsname{\def\PY@tc##1{\textcolor[rgb]{0.73,0.13,0.13}{##1}}}
\expandafter\def\csname PY@tok@mb\endcsname{\def\PY@tc##1{\textcolor[rgb]{0.40,0.40,0.40}{##1}}}
\expandafter\def\csname PY@tok@mf\endcsname{\def\PY@tc##1{\textcolor[rgb]{0.40,0.40,0.40}{##1}}}
\expandafter\def\csname PY@tok@mh\endcsname{\def\PY@tc##1{\textcolor[rgb]{0.40,0.40,0.40}{##1}}}
\expandafter\def\csname PY@tok@mi\endcsname{\def\PY@tc##1{\textcolor[rgb]{0.40,0.40,0.40}{##1}}}
\expandafter\def\csname PY@tok@il\endcsname{\def\PY@tc##1{\textcolor[rgb]{0.40,0.40,0.40}{##1}}}
\expandafter\def\csname PY@tok@mo\endcsname{\def\PY@tc##1{\textcolor[rgb]{0.40,0.40,0.40}{##1}}}
\expandafter\def\csname PY@tok@ch\endcsname{\let\PY@it=\textit\def\PY@tc##1{\textcolor[rgb]{0.25,0.50,0.50}{##1}}}
\expandafter\def\csname PY@tok@cm\endcsname{\let\PY@it=\textit\def\PY@tc##1{\textcolor[rgb]{0.25,0.50,0.50}{##1}}}
\expandafter\def\csname PY@tok@cpf\endcsname{\let\PY@it=\textit\def\PY@tc##1{\textcolor[rgb]{0.25,0.50,0.50}{##1}}}
\expandafter\def\csname PY@tok@c1\endcsname{\let\PY@it=\textit\def\PY@tc##1{\textcolor[rgb]{0.25,0.50,0.50}{##1}}}
\expandafter\def\csname PY@tok@cs\endcsname{\let\PY@it=\textit\def\PY@tc##1{\textcolor[rgb]{0.25,0.50,0.50}{##1}}}

\def\PYZbs{\char`\\}
\def\PYZus{\char`\_}
\def\PYZob{\char`\{}
\def\PYZcb{\char`\}}
\def\PYZca{\char`\^}
\def\PYZam{\char`\&}
\def\PYZlt{\char`\<}
\def\PYZgt{\char`\>}
\def\PYZsh{\char`\#}
\def\PYZpc{\char`\%}
\def\PYZdl{\char`\$}
\def\PYZhy{\char`\-}
\def\PYZsq{\char`\'}
\def\PYZdq{\char`\"}
\def\PYZti{\char`\~}
% for compatibility with earlier versions
\def\PYZat{@}
\def\PYZlb{[}
\def\PYZrb{]}
\makeatother


    % Exact colors from NB
    \definecolor{incolor}{rgb}{0.0, 0.0, 0.5}
    \definecolor{outcolor}{rgb}{0.545, 0.0, 0.0}



    
    % Prevent overflowing lines due to hard-to-break entities
    \sloppy 
    % Setup hyperref package
    \hypersetup{
      breaklinks=true,  % so long urls are correctly broken across lines
      colorlinks=true,
      urlcolor=urlcolor,
      linkcolor=linkcolor,
      citecolor=citecolor,
      }
    % Slightly bigger margins than the latex defaults
    
    \geometry{verbose,tmargin=1in,bmargin=1in,lmargin=1in,rmargin=1in}
    
    

    \begin{document}
    
    
    \maketitle
    
    

    
    \hypertarget{homework}{%
\subsection{Homework}\label{homework}}

    \hypertarget{history-of-probability}{%
\subsubsection{1. History of Probability}\label{history-of-probability}}

It is said that de Mere had been betting that, in four rolls of a die,
at least one six would turn up. He was winning consistently and, to get
more people to play, he changed the game to bet that, in 24 rolls of two
dice, a pair of sixes would turn up. It is claimed that de Mere lost
with 24 and felt that 25 rolls were necessary to make the game
favorable. It was un grand scandale that mathematics was wrong.

    \begin{Verbatim}[commandchars=\\\{\}]
{\color{incolor}In [{\color{incolor}1}]:} \PY{c+c1}{\PYZsh{}1. Write function to sample from die}
        \PY{c+c1}{\PYZsh{}2. Write loop or function to tally sucess vs failure (define sucess?)}
        \PY{c+c1}{\PYZsh{}3. Run 10000 trials}
        \PY{c+c1}{\PYZsh{}4. Calculate probabilities for 24 rolls and 25 rolls}
\end{Verbatim}


    \begin{Verbatim}[commandchars=\\\{\}]
{\color{incolor}In [{\color{incolor}2}]:} \PY{c+c1}{\PYZsh{} m is the number of dice rolls which will change, either 24 or 25}
        \PY{c+c1}{\PYZsh{} I am hardcoding the dice values 1:6 and that we are rolling 2 dice at a time}
        deMere \PY{o}{\PYZlt{}\PYZhy{}} \PY{k+kr}{function}\PY{p}{(}\PY{p}{)}\PY{p}{\PYZob{}}
            a \PY{o}{\PYZlt{}\PYZhy{}} \PY{k+kp}{sample}\PY{p}{(}\PY{l+m}{1}\PY{o}{:}\PY{l+m}{6}\PY{p}{,}\PY{l+m}{2}\PY{p}{,}replace\PY{o}{=}\PY{n+nb+bp}{T}\PY{p}{)}
            \PY{k+kr}{return} \PY{p}{(}a\PY{p}{)}
        \PY{p}{\PYZcb{}}
\end{Verbatim}


    \begin{Verbatim}[commandchars=\\\{\}]
{\color{incolor}In [{\color{incolor}3}]:} \PY{c+c1}{\PYZsh{}This will just duplicate the deMere roll m times}
        rep\PYZus{}deMere \PY{o}{\PYZlt{}\PYZhy{}} \PY{k+kr}{function}\PY{p}{(}m\PY{p}{)}\PY{p}{\PYZob{}}
            die\PYZus{}list \PY{o}{=} \PY{k+kp}{replicate}\PY{p}{(}m\PY{p}{,}deMere\PY{p}{(}\PY{p}{)}\PY{p}{)}
            \PY{k+kr}{return} \PY{p}{(}die\PYZus{}list\PY{p}{)}
        \PY{p}{\PYZcb{}}
\end{Verbatim}


    \begin{Verbatim}[commandchars=\\\{\}]
{\color{incolor}In [{\color{incolor}4}]:} rep\PYZus{}deMere\PY{p}{(}\PY{l+m}{24}\PY{p}{)}
\end{Verbatim}


    \begin{tabular}{llllllllllllllllllllllll}
	 4 & 6 & 4 & 6 & 1 & 1 & 6 & 3 & 5 & 6 & ⋯ & 2 & 3 & 3 & 5 & 4 & 3 & 3 & 6 & 3 & 4\\
	 3 & 4 & 6 & 4 & 3 & 5 & 3 & 4 & 5 & 6 & ⋯ & 6 & 2 & 1 & 4 & 4 & 3 & 1 & 5 & 5 & 3\\
\end{tabular}


    
    \begin{Verbatim}[commandchars=\\\{\}]
{\color{incolor}In [{\color{incolor}5}]:} \PY{c+c1}{\PYZsh{} m is number of rolls}
        \PY{c+c1}{\PYZsh{} generates all the dice rolls, and checks to see if each die in the pair are 6\PYZsq{}s}
        \PY{c+c1}{\PYZsh{} then reduces the vector of size m to a single TRUE or FALSE by checking to see}
        \PY{c+c1}{\PYZsh{} if a pair of 6s was rolled at least once}
        pair\PYZus{}of\PYZus{}6s \PY{o}{\PYZlt{}\PYZhy{}}\PY{k+kr}{function}\PY{p}{(}m\PY{p}{)}\PY{p}{\PYZob{}}
            \PY{k+kr}{return}\PY{p}{(}\PY{k+kp}{is.element}\PY{p}{(}\PY{k+kc}{TRUE}\PY{p}{,} \PY{p}{(}\PY{k+kp}{apply}\PY{p}{(}rep\PYZus{}deMere\PY{p}{(}m\PY{p}{)}\PY{p}{,}\PY{l+m}{2}\PY{p}{,}\PY{k+kr}{function}\PY{p}{(}r\PY{p}{)} r\PY{p}{[}\PY{l+m}{1}\PY{p}{]} \PY{o}{==} \PY{l+s}{\PYZdq{}}\PY{l+s}{6\PYZdq{}} \PY{o}{\PYZam{}} r\PY{p}{[}\PY{l+m}{2}\PY{p}{]} \PY{o}{==} \PY{l+s}{\PYZdq{}}\PY{l+s}{6\PYZdq{}}\PY{p}{)}\PY{p}{)}\PY{p}{)}\PY{p}{)}
        \PY{p}{\PYZcb{}}                                   
        pair\PYZus{}of\PYZus{}6s\PY{p}{(}\PY{l+m}{24}\PY{p}{)}
\end{Verbatim}


    FALSE

    
    \hypertarget{probability-with-24-rolls}{%
\subsubsection{Probability with 24
rolls}\label{probability-with-24-rolls}}

    \begin{Verbatim}[commandchars=\\\{\}]
{\color{incolor}In [{\color{incolor}6}]:} rolling24 \PY{o}{\PYZlt{}\PYZhy{}} \PY{k+kp}{replicate}\PY{p}{(}\PY{l+m}{10000}\PY{p}{,}pair\PYZus{}of\PYZus{}6s\PY{p}{(}\PY{l+m}{24}\PY{p}{)}\PY{p}{)}
        \PY{k+kp}{length}\PY{p}{(}\PY{k+kp}{which}\PY{p}{(}rolling24 \PY{o}{==} \PY{l+s}{\PYZdq{}}\PY{l+s}{TRUE\PYZdq{}}\PY{p}{)}\PY{p}{)}\PY{o}{/}\PY{k+kp}{length}\PY{p}{(}rolling24\PY{p}{)}
\end{Verbatim}


    0.4974

    
    \hypertarget{probability-with-25-rolls}{%
\subsubsection{Probability with 25
rolls}\label{probability-with-25-rolls}}

    \begin{Verbatim}[commandchars=\\\{\}]
{\color{incolor}In [{\color{incolor}7}]:} rolling25 \PY{o}{\PYZlt{}\PYZhy{}} \PY{k+kp}{replicate}\PY{p}{(}\PY{l+m}{10000}\PY{p}{,}pair\PYZus{}of\PYZus{}6s\PY{p}{(}\PY{l+m}{25}\PY{p}{)}\PY{p}{)}
        \PY{k+kp}{length}\PY{p}{(}\PY{k+kp}{which}\PY{p}{(}rolling25 \PY{o}{==} \PY{l+s}{\PYZdq{}}\PY{l+s}{TRUE\PYZdq{}}\PY{p}{)}\PY{p}{)}\PY{o}{/}\PY{k+kp}{length}\PY{p}{(}rolling25\PY{p}{)}
\end{Verbatim}


    0.5054

    
    \hypertarget{addition-rule}{%
\subsection{2. Addition rule}\label{addition-rule}}

\hypertarget{find-the-probabilities-using-the-table}{%
\paragraph{Find the probabilities using the
table}\label{find-the-probabilities-using-the-table}}

\begin{quote}
\begin{enumerate}
\def\labelenumi{\arabic{enumi}.}
\tightlist
\item
  Type O or AB.
\end{enumerate}
\end{quote}

\begin{quote}
\begin{enumerate}
\def\labelenumi{\arabic{enumi}.}
\setcounter{enumi}{1}
\tightlist
\item
  Type A or AB.
\end{enumerate}
\end{quote}

\begin{quote}
\begin{enumerate}
\def\labelenumi{\arabic{enumi}.}
\setcounter{enumi}{2}
\tightlist
\item
  Type AB or Rh negative.
\end{enumerate}
\end{quote}

\begin{quote}
\begin{enumerate}
\def\labelenumi{\arabic{enumi}.}
\setcounter{enumi}{3}
\tightlist
\item
  Type O and Negative.
\end{enumerate}
\end{quote}

\begin{quote}
\begin{enumerate}
\def\labelenumi{\arabic{enumi}.}
\setcounter{enumi}{4}
\tightlist
\item
  Type AB
\end{enumerate}
\end{quote}

    \begin{longtable}[]{@{}llllll@{}}
\toprule
Type & O & A & B & AB & Total\tabularnewline
\midrule
\endhead
Positive & 163 & 662 & 1513 & 1603 & 3941\tabularnewline
Negative & 224 & 933 & 2400 & 2337 & 5894\tabularnewline
Total & 387 & 1595 & 3913 & 3940 & 9835\tabularnewline
\bottomrule
\end{longtable}

    \begin{Verbatim}[commandchars=\\\{\}]
{\color{incolor}In [{\color{incolor}8}]:} prob1 \PY{o}{=} \PY{l+m}{387}\PY{o}{/}\PY{l+m}{9835} \PY{o}{+} \PY{l+m}{3940}\PY{o}{/}\PY{l+m}{9835}
        prob1
        
        prob2 \PY{o}{=} \PY{l+m}{1595}\PY{o}{/}\PY{l+m}{9835} \PY{o}{+} \PY{l+m}{3940}\PY{o}{/}\PY{l+m}{9835}
        prob2
        
        prob3 \PY{o}{=} \PY{l+m}{3940}\PY{o}{/}\PY{l+m}{9835} \PY{o}{+} \PY{l+m}{5894}\PY{o}{/}\PY{l+m}{9835} \PY{o}{\PYZhy{}} \PY{p}{(}\PY{l+m}{2337}\PY{o}{/}\PY{l+m}{9835}\PY{p}{)}
        prob3
        
        prob4 \PY{o}{=} \PY{l+m}{224}\PY{o}{/}\PY{l+m}{9835}
        prob4
        
        prob5 \PY{o}{=} \PY{l+m}{3940}\PY{o}{/}\PY{l+m}{9835}
        prob5
\end{Verbatim}


    0.4399593289273

    
    0.562785968479919

    
    0.762277580071174

    
    0.0227758007117438

    
    0.400610066090493

    
    \hypertarget{multiplication-rule}{%
\subsubsection{3. Multiplication Rule}\label{multiplication-rule}}

\hypertarget{from-this-exercises-if-i-roll-5-dice-what-is-the-chance-of-getting-all-sixes-what-is-the-chance-of-getting-no-sixes}{%
\subparagraph{From this exercises: If I roll 5 dice, what is the chance
of getting all sixes? What is the chance of getting no
sixes?}\label{from-this-exercises-if-i-roll-5-dice-what-is-the-chance-of-getting-all-sixes-what-is-the-chance-of-getting-no-sixes}}

Write a simulation in R to obtain the probabilities for each of these
two exercises running 10000 trials

Hint.

Remember that the change to get all 6 in 5 rolls is \((1/6)^5\)

and to get no sixes is \((5/6)^5\)

    \begin{Verbatim}[commandchars=\\\{\}]
{\color{incolor}In [{\color{incolor}9}]:} \PY{c+c1}{\PYZsh{}All sixes}
        \PY{c+c1}{\PYZsh{}1. Write function to sample from die}
        \PY{c+c1}{\PYZsh{}2. Write function to determine if in each 5 rolls we get all sixes (6*5) }
        \PY{c+c1}{\PYZsh{}3. Run 10000 trials \PYZhy{} Hint check sapply function}
        \PY{c+c1}{\PYZsh{}4. Calculate probability}
        
        \PY{c+c1}{\PYZsh{}No sixes}
        
        \PY{c+c1}{\PYZsh{}1. Write function to sample from die \PYZhy{} \PYZhy{} Hint check sapply function \PYZhy{}}
        \PY{c+c1}{\PYZsh{}2. Write function to determine if in each 5 rolls we get no sixes (6*5)}
        \PY{c+c1}{\PYZsh{}3. Run 10000 trials }
        \PY{c+c1}{\PYZsh{}4. Calculate probability}
\end{Verbatim}


    \begin{Verbatim}[commandchars=\\\{\}]
{\color{incolor}In [{\color{incolor}10}]:} die\PYZus{}roll \PY{o}{\PYZlt{}\PYZhy{}} \PY{k+kr}{function}\PY{p}{(}n\PY{p}{)}\PY{p}{\PYZob{}}
             \PY{k+kr}{return} \PY{p}{(}\PY{k+kp}{sample}\PY{p}{(}\PY{l+m}{1}\PY{o}{:}\PY{l+m}{6}\PY{p}{,}n\PY{p}{,}replace \PY{o}{=} \PY{n+nb+bp}{T}\PY{p}{)}\PY{p}{)}
         \PY{p}{\PYZcb{}}
         \PY{c+c1}{\PYZsh{}6\PYZca{}5}
\end{Verbatim}


    \hypertarget{probability-of-rolling-all-6s}{%
\subsubsection{Probability of rolling all
6's}\label{probability-of-rolling-all-6s}}

    \begin{Verbatim}[commandchars=\\\{\}]
{\color{incolor}In [{\color{incolor}11}]:} x \PY{o}{\PYZlt{}\PYZhy{}}\PY{k+kp}{replicate}\PY{p}{(}\PY{l+m}{10000}\PY{p}{,}die\PYZus{}roll\PY{p}{(}\PY{l+m}{5}\PY{p}{)}\PY{p}{)}
         \PY{c+c1}{\PYZsh{}x1 \PYZlt{}\PYZhy{} apply(x,2, function(y) print(prod(y)) \PYZam{} prod(y)==7776)}
         x1 \PY{o}{\PYZlt{}\PYZhy{}} \PY{k+kp}{apply}\PY{p}{(}x\PY{p}{,}\PY{l+m}{2}\PY{p}{,} \PY{k+kr}{function}\PY{p}{(}y\PY{p}{)} \PY{p}{(}\PY{k+kp}{prod}\PY{p}{(}y\PY{p}{)}\PY{o}{==}\PY{l+m}{7776}\PY{p}{)}\PY{p}{)}
         \PY{c+c1}{\PYZsh{}x1}
         \PY{k+kp}{length}\PY{p}{(}\PY{k+kp}{which}\PY{p}{(}x1\PY{o}{==}\PY{l+s}{\PYZdq{}}\PY{l+s}{TRUE\PYZdq{}}\PY{p}{)}\PY{p}{)}\PY{o}{/}\PY{k+kp}{length}\PY{p}{(}x1\PY{p}{)}
\end{Verbatim}


    2e-04

    
    \hypertarget{probability-of-rolling-no-6s}{%
\subsubsection{Probability of rolling NO
6's}\label{probability-of-rolling-no-6s}}

    \begin{Verbatim}[commandchars=\\\{\}]
{\color{incolor}In [{\color{incolor}12}]:} x \PY{o}{\PYZlt{}\PYZhy{}}\PY{k+kp}{replicate}\PY{p}{(}\PY{l+m}{10000}\PY{p}{,}die\PYZus{}roll\PY{p}{(}\PY{l+m}{5}\PY{p}{)}\PY{p}{)}
         \PY{c+c1}{\PYZsh{}x1 \PYZlt{}\PYZhy{} apply(x,2, function(y) prod(y) == 7776)}
         \PY{c+c1}{\PYZsh{}x1 \PYZlt{}\PYZhy{} apply(x,2, function(y) is.element(\PYZdq{}6\PYZdq{},y) \PYZam{}print(y))}
         x1 \PY{o}{\PYZlt{}\PYZhy{}} \PY{k+kp}{apply}\PY{p}{(}x\PY{p}{,}\PY{l+m}{2}\PY{p}{,} \PY{k+kr}{function}\PY{p}{(}y\PY{p}{)} \PY{k+kp}{is.element}\PY{p}{(}\PY{l+s}{\PYZdq{}}\PY{l+s}{6\PYZdq{}}\PY{p}{,}y\PY{p}{)}\PY{p}{)}
         \PY{c+c1}{\PYZsh{}x1}
         x2 \PY{o}{\PYZlt{}\PYZhy{}} \PY{k+kp}{length}\PY{p}{(}\PY{k+kp}{which}\PY{p}{(}x1 \PY{o}{==} \PY{l+s}{\PYZdq{}}\PY{l+s}{FALSE\PYZdq{}}\PY{p}{)}\PY{p}{)}\PY{o}{/}\PY{k+kp}{length}\PY{p}{(}x1\PY{p}{)}
         x2
\end{Verbatim}


    0.3967

    
    \hypertarget{conditional-probability}{%
\subsubsection{4. Conditional
Probability}\label{conditional-probability}}

Consider a family that has three children. We are interested in the
children's genders. Our sample space is
S=\{(G,G,G),(G,G,B),(G,B,G),(G,B,B),(B,G,G),(B,G,B),(B,B,G),(B,B,B)\}.
Also assume that all eight possible outcomes are equally likely.

\begin{enumerate}
\def\labelenumi{\arabic{enumi}.}
\item
  What is the probability that the three children are girls given that
  the first child is a girl?
\item
  What is the probability that At least two children are boys given that
  the first child is a boy?
\end{enumerate}

    P(AandB)/ P(B)

    P(A) = three girls P(B) = first child = girl

    \begin{Verbatim}[commandchars=\\\{\}]
{\color{incolor}In [{\color{incolor}13}]:} probA \PY{o}{=} \PY{l+m}{1}\PY{o}{/}\PY{l+m}{8}
         probB \PY{o}{=} \PY{l+m}{1}\PY{o}{/}\PY{l+m}{2}
         probA\PYZus{}and\PYZus{}B \PY{o}{=} \PY{l+m}{1}\PY{o}{/}\PY{l+m}{8}
         answer1 \PY{o}{=} probA\PYZus{}and\PYZus{}B\PY{o}{/}probB
         answer1
\end{Verbatim}


    0.25

    
    \begin{Verbatim}[commandchars=\\\{\}]
{\color{incolor}In [{\color{incolor}14}]:} probA \PY{o}{=} \PY{l+m}{1}\PY{o}{/}\PY{l+m}{2}
         probB \PY{o}{=} \PY{l+m}{1}\PY{o}{/}\PY{l+m}{2}
         probA\PYZus{}and\PYZus{}B \PY{o}{=} \PY{l+m}{3}\PY{o}{/}\PY{l+m}{8}
         answer2 \PY{o}{=} probA\PYZus{}and\PYZus{}B\PY{o}{/}probB
         answer2
\end{Verbatim}


    0.75

    

    % Add a bibliography block to the postdoc
    
    
    
    \end{document}
